
\documentclass[
	% -- opções da classe memoir --
	article,			% indica que é um artigo acadêmico
	11pt,				% tamanho da fonte
	oneside,			% para impressão apenas no verso. Oposto a twoside
	a4paper,			% tamanho do papel. 
	% -- opções da classe abntex2 --
	%chapter=TITLE,		% títulos de capítulos convertidos em letras maiúsculas
	%section=TITLE,		% títulos de seções convertidos em letras maiúsculas
	%subsection=TITLE,	% títulos de subseções convertidos em letras maiúsculas
	%subsubsection=TITLE % títulos de subsubseções convertidos em letras maiúsculas
	% -- opções do pacote babel --
	english,			% idioma adicional para hifenização
	brazil,				% o último idioma é o principal do documento
	sumario=tradicional
	]{abntex2}

\newcommand{\matlabCodePath}{/home/clifte/git/Mestrado/Matlab/} 
% ---
% PACOTES
% ---

% ---
% Pacotes fundamentais 
% ---
\usepackage{lmodern}			% Usa a fonte Latin Modern
\usepackage[T1]{fontenc}		% Selecao de codigos de fonte.
\usepackage[utf8]{inputenc}		% Codificacao do documento (conversão automática dos acentos)
\usepackage{indentfirst}		% Indenta o primeiro parágrafo de cada seção.
\usepackage{nomencl} 			% Lista de simbolos
\usepackage{color}				% Controle das cores
\usepackage{graphicx}			% Inclusão de gráficos
\usepackage{microtype} 			% para melhorias de justificação

\usepackage{amsmath} 			% Equações
\usepackage{graphicx}
\usepackage{caption}
\usepackage{subcaption}
\usepackage{tikz}
\usepackage{listings}
\usepackage[]{mcode}
\usepackage{listingsutf8}
\usepackage{epstopdf}
% ---
		
% ---
% Pacotes adicionais, usados apenas no âmbito do Modelo Canônico do abnteX2
% ---
\usepackage{lipsum}				% para geração de dummy text
% ---
		
% ---
% Pacotes de citações
% ---
\usepackage[brazilian,hyperpageref]{backref}	 % Paginas com as citações na bibl
\usepackage[alf]{abntex2cite}	% Citações padrão ABNT


% ---

% ---
% Configurações do pacote backref
% Usado sem a opção hyperpageref de backref
\renewcommand{\backrefpagesname}{Citado na(s) página(s):~}
% Texto padrão antes do número das páginas
\renewcommand{\backref}{}
% Define os textos da citação
\renewcommand*{\backrefalt}[4]{
	\ifcase #1 %
		Nenhuma citação no texto.%
	\or
		Citado na página #2.%
	\else
		Citado #1 vezes nas páginas #2.%
	\fi}%
% ---

% ---
% Informações de dados para CAPA e FOLHA DE ROSTO
% ---
\titulo{Relatório de Trabalho 2}
\autor{David Clifte da Silva Vieira}
\local{Brasil}
\data{2014, 21 de outubro}
% ---

% ---
% Configurações de aparência do PDF final

% alterando o aspecto da cor azul
\definecolor{blue}{RGB}{41,5,195}

% informações do PDF
\makeatletter
\hypersetup{
     	%pagebackref=true,
		pdftitle={\@title}, 
		pdfauthor={\@author},
    	pdfsubject={Modelo de artigo científico com abnTeX2},
	    pdfcreator={LaTeX with abnTeX2},
		pdfkeywords={abnt}{latex}{abntex}{abntex2}{atigo científico}, 
		colorlinks=true,       		% false: boxed links; true: colored links
    	linkcolor=blue,          	% color of internal links
    	citecolor=blue,        		% color of links to bibliography
    	filecolor=magenta,      		% color of file links
		urlcolor=blue,
		bookmarksdepth=4
}
\makeatother
% --- 

% ---
% compila o indice
% ---
\makeindex
% ---

% ---
% Altera as margens padrões
% ---
\setlrmarginsandblock{3cm}{3cm}{*}
\setulmarginsandblock{3cm}{3cm}{*}
\checkandfixthelayout
% ---

% --- 
% Espaçamentos entre linhas e parágrafos 
% --- 

% O tamanho do parágrafo é dado por:
\setlength{\parindent}{1.3cm}

% Controle do espaçamento entre um parágrafo e outro:
\setlength{\parskip}{0.2cm}  % tente também \onelineskip

% Espaçamento simples
\SingleSpacing

% ----
% Início do documento
% ----
\begin{document}
% Retira espaço extra obsoleto entre as frases.


% ----------------------------------------------------------
% ELEMENTOS PRÉ-TEXTUAIS
% ----------------------------------------------------------

%---
%
% Se desejar escrever o artigo em duas colunas, descomente a linha abaixo
% e a linha com o texto ``FIM DE ARTIGO EM DUAS COLUNAS''.
% \twocolumn[    		% INICIO DE ARTIGO EM DUAS COLUNAS
%
%---
% página de titulo
\maketitle
\frenchspacing 


% ]  				% FIM DE ARTIGO EM DUAS COLUNAS
% ---

% ----------------------------------------------------------
% ELEMENTOS TEXTUAIS
% ----------------------------------------------------------
\textual

% ----------------------------------------------------------
% Introdução
% ----------------------------------------------------------
\section*{Introdução}
\addcontentsline{toc}{section}{Introdução}
Este trabalho apresenta o resultados obtidos durante o desenvolvimento da
segunda lista de exercícios propostos pelo professor Ajalmar Roccha no curso de
mestrado em Ciências da Computação do IFCE. Parte do código fonte é exibido na
forma de Apêndice ao fim do trabalho.

\section*{RBF - Radial Basis Function}

\subsection{Teorema de Cover} As redes RBF tem como princípio a transformação
não linear das entradas para um espaço de alta dimensionalidade. Segundo o
teorema de Cover sobre a separabilidade de padrões, nesse novo espaço de alta
dimensionalidade a probabilidade de existir um hiperplano que posso separar
linearmente os dados é maior.

\begin{citacao}
Um problema complexo de classificação de padrões disposto não linearmente em um
espaço de alta dimensão tem maior probabilidade de ser linearmente separável do
que em um espaço de baixa dimensionalidade.(Cover, 1965) 
\end{citacao}

\begin{align}
P(N,m_1) = \left(\frac{1}{2}\right)^{N-1}   
\sum_{m=0}^{m_1-1}   
\binom{N-1}{m}
\label{eq:teoSepCover}
\end{align}

Na equação \ref{eq:teoSepCover}, citada no trabalho de Cover temos a
probabilidade de um padrão $x$ ser separado linearmente de $\chi=\{x_i\}_{i=1}^{N}$ após passar por uma
transformação através de uma função oculta $\phi(x)$ que mapeia o padrão $x$ em
um espaço de maior dimensionalidade.
Podemos perceber que a medida que o número de dimensões $m_1$ aumentam para uma
determinado número $N$ padrões a probabilidade resultante tende a 1.
Do teorema de cover podemos deduzir também que:
\begin{citacao}
O número máximo de padrões que são linearmente separáveis em um espaço de
dimensão $m_1$ é igual a $2m_1$. (Cover, 1965).
\end{citacao}

\subsection{Mapeamento da função oculta}
Um mapeamento não linear é utilizado para realizar uma tranfomação nos dados de
classificação de forma que ele possa ser resolvido linearmente. Como visto na
sessão anterior, Cover resolveu este problema ao aumentar o número de dimensões
dos padrões.

O mapeamento é feito através do vetor $\phi(x)$.
\begin{align}
\phi(x) = [ \phi_1(x), \phi_2(x), \ldots, \phi_{m1}(x) ]^t 
\label{eq:phiVec}
\end{align}
Dado um vetor $x$ oriundo do espaço de entrada de dimensão $m_0$. O vetor
$\phi(x)$ realiza o mapeamento de $x$ em novo espaço de dimensão $m_1$. Como
representado na equação \ref{eq:phiVec}, $\phi(x)$ é um conjunto de funções
ocultas $\{\phi_i(x)\}^{m_1}_{i=1}$ que mapeiam o vetor $x$ no novo espaço
denominado espaço oculto ou espaço de caracaterísticas. É esperado que no espaço
de características seja possível realizar a classificação dos dados de forma
linear. O hiperplano que separa as classes é definido por
\begin{align}
W^T\varphi(x)=0
\end{align} 

Desta forma temos que para $x \in \chi_1 $, $W^T\varphi(x)>0$, onde $W$ é um
vetor com os termos da combinação das funções de base radial.

\subsection{A Rede RBF}
Uma rede RBF(Radial Basis function), da mesma forma que rede MLP e ELM é
composta por três camadas.
Uma camada de entrada, camada oculta e camada de saída. A principal diferença entre
este modelo e os outros está na camda oculta que apresenta como função de
ativação uma função de base radial.

Uma função de base radial é caracterizada pelo valor aumentar ou
diminuir de acordo com a distancia ao ponto central da função. A função
gaussiana geralmente é utilizadas em redes RBFs, ela se caracteriza por valores
próximos da média apresentarem valores próximos a 1 e valores distantes da média
apresentam valores próximos a 0. Finalmente a rede RBF pode ser representada
da seguinte maneira.

A resposta a uma entrada $x$ em uma rede RBF pode ser representada como:
\begin{align}
F(X)=\sum^{N}_{i=1}W_i\phi(\|x-x_i\|)
\end{align}

\subsection{Treinamento da rede}
O treinamento de uma rede RBF possui duas fases. A princípio é necessário
definir os parâmetros das funções de base radial e posteriormente os pesos
utilizados nos neurônios de saída. Uma técnica geralmente utilizada para a
definição dos parâmetros das funções de base radial é o K-means, nele K
vetores médios pertecentes aos dados de treinamentos são escolhidos como centro
das funções de base radial(Bishop, 1995). Uma outra abordagem possível para a
definição dos centros das funções é realizar a subamostragem dos vetores de
entrada. Após a definição das funções de base radial resta apenas o calculo dos
pesos da camada de saída. Como dito anteriormente este problema se resume a
otimização de um classificador linear e pode ser obtido com a tecnica simples
da pseudo inversa (Haykin, 2008).

\subsection{Exemplo de aplicação de uma rede RBF}


% --- 
% Finaliza a parte no bookmark do PDF, para que se inicie o bookmark na raiz
% ---
\bookmarksetup{startatroot}% 
% ---


\begin{citacao}

\end{citacao}


% ]  				% FIM DE ARTIGO EM DUAS COLUNAS
% ---

% ----------------------------------------------------------
% Referências bibliográficas
% ----------------------------------------------------------
\bibliography{pdi}
Bishop, C.M., 1995, "Neural Networks for Pattern Recognition" (first ed.),
Oxford University Press Inc., New York, USA, 482 p.
 


 
% ----------------------------------------------------------
% Glossário
% ----------------------------------------------------------
%
% Há diversas soluções prontas para glossário em LaTeX. 
% Consulte o manual do abnTeX2 para obter sugestões.
%
%\glossary

% ----------------------------------------------------------
% Apêndices
% ----------------------------------------------------------

% ---
% Inicia os apêndices 
% ---
\begin{apendicesenv} 

% ----------------------------------------------------------
\chapter{Código fonte: Translação, Rotação e Redimensionamento}
\label{apend:transRotRed}

	
 	\lstset{extendedchars=true,inputencoding=utf8/latin1}
 	\lstinputlisting[frame=single,
 					 numbers=left]{\matlabCodePath
 	Trabalho_PDI/q11/topico11.m}



\end{apendicesenv}
% ---

% ----------------------------------------------------------
% Anexos
% ----------------------------------------------------------
\cftinserthook{toc}{AAA}


% ---
% Inicia os anexos
% ---
%\anexos
%\begin{anexosenv}
%\chapter{}
%\end{anexosenv}

\end{document}

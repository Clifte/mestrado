% abtex2-modelo-artigo.tex, v-1.9.2 laurocesar
% Copyright 2012-2014 by abnTeX2 group at http://abntex2.googlecode.com/ 
%

% ------------------------------------------------------------------------
% ------------------------------------------------------------------------
% abnTeX2: Modelo de Artigo Acadêmico em conformidade com
% ABNT NBR 6022:2003: Informação e documentação - Artigo em publicação 
% periódica científica impressa - Apresentação
% ------------------------------------------------------------------------
% ------------------------------------------------------------------------

\documentclass[
	% -- opções da classe memoir --
	article,			% indica que é um artigo acadêmico
	11pt,				% tamanho da fonte
	oneside,			% para impressão apenas no verso. Oposto a twoside
	a4paper,			% tamanho do papel. 
	% -- opções da classe abntex2 --
	%chapter=TITLE,		% títulos de capítulos convertidos em letras maiúsculas
	%section=TITLE,		% títulos de seções convertidos em letras maiúsculas
	%subsection=TITLE,	% títulos de subseções convertidos em letras maiúsculas
	%subsubsection=TITLE % títulos de subsubseções convertidos em letras maiúsculas
	% -- opções do pacote babel --
	english,			% idioma adicional para hifenização
	brazil,				% o último idioma é o principal do documento
	sumario=tradicional
	]{abntex2}

 
% ---
% PACOTES
% ---

% ---
% Pacotes fundamentais 
% ---
\usepackage{lmodern}			% Usa a fonte Latin Modern
\usepackage[T1]{fontenc}		% Selecao de codigos de fonte.
\usepackage[utf8]{inputenc}		% Codificacao do documento (conversão automática dos acentos)
\usepackage{indentfirst}		% Indenta o primeiro parágrafo de cada seção.
\usepackage{nomencl} 			% Lista de simbolos
\usepackage{color}				% Controle das cores
\usepackage{graphicx}			% Inclusão de gráficos
\usepackage{microtype} 			% para melhorias de justificação

\usepackage{amsmath} 			% Equações
\usepackage{graphicx}
\usepackage{caption}
\usepackage{subcaption}
\usepackage{tikz}
\usepackage{listings}
\usepackage[]{mcode}
\usepackage{listingsutf8}
\usepackage{epstopdf}
% ---
		
% ---
% Pacotes adicionais, usados apenas no âmbito do Modelo Canônico do abnteX2
% ---
\usepackage{lipsum}				% para geração de dummy text
% ---
		
% ---
% Pacotes de citações
% ---
\usepackage[brazilian,hyperpageref]{backref}	 % Paginas com as citações na bibl
\usepackage[alf]{abntex2cite}	% Citações padrão ABNT
% ---

% ---
% Configurações do pacote backref
% Usado sem a opção hyperpageref de backref
\renewcommand{\backrefpagesname}{Citado na(s) página(s):~}
% Texto padrão antes do número das páginas
\renewcommand{\backref}{}
% Define os textos da citação
\renewcommand*{\backrefalt}[4]{
	\ifcase #1 %
		Nenhuma citação no texto.%
	\or
		Citado na página #2.%
	\else
		Citado #1 vezes nas páginas #2.%
	\fi}%
% ---

% ---
% Informações de dados para CAPA e FOLHA DE ROSTO
% ---
\titulo{GPGPU aplicada ao reconhecimento ótico de caracteres utilizando
técnica de n vizinho mais próximos.}
\autor{David Clifte\thanks{cliftedavid@gmail.com}
\and Gustavo Siebra\thanks{gustavosiebra@gmail.com}} 
\local{Brasil}
\data{\today}
% ---

% ---
% Configurações de aparência do PDF final

% alterando o aspecto da cor azul
\definecolor{blue}{RGB}{41,5,195}

% informações do PDF
\makeatletter
\hypersetup{
     	%pagebackref=true,
		pdftitle={\@title}, 
		pdfauthor={\@author},
    	pdfsubject={Modelo de artigo científico com abnTeX2},
	    pdfcreator={LaTeX with abnTeX2},
		pdfkeywords={abnt}{latex}{abntex}{abntex2}{atigo científico}, 
		colorlinks=true,       		% false: boxed links; true: colored links
    	linkcolor=blue,          	% color of internal links
    	citecolor=blue,        		% color of links to bibliography
    	filecolor=magenta,      		% color of file links
		urlcolor=blue,
		bookmarksdepth=4
}
\makeatother
% --- 

% ---
% compila o indice
% ---
\makeindex
% ---

% ---
% Altera as margens padrões
% ---
\setlrmarginsandblock{3cm}{3cm}{*}
\setulmarginsandblock{3cm}{3cm}{*}
\checkandfixthelayout
% ---

% --- 
% Espaçamentos entre linhas e parágrafos 
% --- 

% O tamanho do parágrafo é dado por:
\setlength{\parindent}{1.3cm}

% Controle do espaçamento entre um parágrafo e outro:
\setlength{\parskip}{0.2cm}  % tente também \onelineskip

% Espaçamento simples
\SingleSpacing

% ----
% Início do documento
% ----
\begin{document}

% Retira espaço extra obsoleto entre as frases.
\frenchspacing 

% ----------------------------------------------------------
% ELEMENTOS PRÉ-TEXTUAIS
% ----------------------------------------------------------

%---
%
% Se desejar escrever o artigo em duas colunas, descomente a linha abaixo
% e a linha com o texto ``FIM DE ARTIGO EM DUAS COLUNAS''.
% \twocolumn[    		% INICIO DE ARTIGO EM DUAS COLUNAS
%
%---
% página de titulo
\maketitle

% resumo em português
\begin{resumoumacoluna}
Este artigo descreve o uso da rede neural, visão computacional e  paralelismo
para analisar o custo  das imagens recebidas de um scanner ou outra forma de
digitalização para reconhecer o dígito manuscrito e classificá-lo de 0-9, ou
indefinido, de acordo com o valor correspondente. As principais características
do sistema são: Controlar o grau de acerto, um sistema de autoaprendizagem de
fácil implementação e modularizado.
 
 \vspace{\onelineskip}
 
 \noindent
 \textbf{Palavras-chave}: OCR, cuda, rede neural, visão computacional.
\end{resumoumacoluna}


% ----------------------------------------------------------
% Introdução
% ----------------------------------------------------------
\section*{Introdução}
\addcontentsline{toc}{section}{Introdução}

O reconhecimento ótico de caracteres (OCR) permite que uma máquina possa
reconhecer automaticamente um caractere através de um mecanismo óptico. As
tentativas da engenharia em reconhecer caracteres impressos, ou manuscritos,
iniciaram antes da Segunda Guerra Mundial, mas isso não foi possível até a
década de 50, quando a associação dos Bancos e a Indústria dos serviços
financeiros criaram fundos para a pesquisa e desenvolvimento da tecnologia
[1].Diante da necessidade de uma abordagem mais simples aos temas de visão
computacional e algoritmos de inteligência computacional, ambos tem custo
computacional alto devido as rotinas necessárias para aplicações com OCR, foi
desenvolvido um sistema para diminuir o custo com o suporte de placas gráficas
GPU.



Computadores podem executar muitas operações com um tempo consideravelmente
menor que os humanos poderiam fazer. Contudo, nem sempre, essa rapidez é a
melhor escolha para resolver um problema. Muitas tarefas com as quais os
computadores falham consideravelmente os humanos fazem melhor. Muitas dessas
tarefas, nas quais os computadores perdem estão relacionadas à natureza
interpretativa e de multi-processamento do cérebro. Uma maneira simples de
caracterizar bem a diferença entre o computador e o Homem seria comparar o
computador, que é uma máquina serial, com o nosso cérebro, que é altamente
paralelo e possui como característica principal a capacidade de aprender coisas
[3].

A expressão rede neural, como é normalmente usada, é empregada de forma
incompleta. Computadores tentam simular redes neurais biológicas através da
implementação de redes neurais artificiais. Contudo, grande parte das
publicações usam o termo rede neural ao invés do termo rede neural artificial
(RNA).

Uma rede neural artificial é um paradigma baseado na forma biológica do sistema
nervoso e de como ele processa a informação. O elemento mais importante desse
paradigma é a forma como a informação é processada. A rede é composta por um
grande número de elementos de processamento interconectados atuando juntos para
resolver uma tarefa. RNAs, bem como pessoas, aprendem baseados em experiências
passadas e ou exemplos. A rede normalmente é configurada para uma aplicação
específica, como reconhecimento de padrões ou classificação de dados, através de
um processo de treinamento. O sistema de aprendizagem biológico baseia-se em
ajustar as conexões sinápticas que existem entre os neurônios.

Nesse projeto foi utlizado a rede RBF que tem como princípio a transformação não
linear das entradas para um espaço de alta dimensionalidade. Segundo o teorema
de Cover sobre a separabilidade de padrões, nesse novo espaço de alta
dimensionalidade a probabilidade de existir um hiperplano que posso separar
linearmente os dados é maior.

\section{Pré-processamento}

\subsection{Limiarização}
        A segmentação de imagens é um processo que tipicamente particiona o
domínio espacial de uma imagem em subconjuntos mutuamente exclusivos,
chamados regiões, onde cada região é uniforme e homogênea com respeito a algumas
propriedades como tom ou textura e cujos valores diferem, em alguns aspectos e
significados, das propriedades de cada região vizinha.
       A limiarização de uma imagem digital é um método que se baseia no
histograma da imagem, buscando encontrar regiões bem definidas, afim de poder
efetuar a divisão da imagem em objetos ou regiões. A continuidade dos níveis de
cinza é a primitiva de maior valor na segmentação por região. Assim, a limiarização
efetua a subdivisão da imagem em função das regiões realmente significativas
contidas no seu histograma [5].
        A limiarização global usa um limiar fixo para todos os pixels na imagem e, por
esta razão, resultados realmente satisfatórios serão obtidos quando o histograma de
distribuição de níveis de cinza contém cumes distintos e separados que
correspondem aos objetos e ao fundo. Isto conduz a conclusão que um limiar mais
local deve ser usado. 
• A imagem original é dividida em sub-imagens;\newline
• Um limiar é determinado independentemente para cada sub-imagem;\newline
• Se um limiar não puder ser determinado para alguma sub-imagem, ele
  pode ser interpolado à partir dos limiares determinados nas sub-imagens
  vizinhas;\newline
• Cada sub-imagem é então processada utilizando seu limiar local.
     Uma definição de um limiar para técnicas adaptativas pode ser denotada
como sendo:

\begin{align}
T=T[x,y,p(x,y),f(x,y)]
\end{align}
\\Sendo f(x,y) é o nível de cinza do ponto (x,y) na imagem original, e p(x,y) é
alguma propriedade local deste ponto. Percebe-se que o limiar T não depende
apenas do nível de cinza do ponto. É necessário atenção especial ao fator p(x,y) que
é descrito como uma propriedade do ponto e que é, um dos mais importantes
componentes no cálculo do limiar para um certo ponto [MIL98]. Para que a influência
de ruído ou iluminação possa ser levada em conta, o cálculo dessa propriedade é
baseado no ambiente em que o ponto em questão está inserido. Um exemplo de
propriedade é a média dos níveis de cinza em uma vizinhança pré-estabelecida, na
qual o centro é o ponto em análise. Este método produzirá os resultados desejados



\subsection{Detecção de componentes conexos}
      O resultado obtido na localização de componentes normalmente apresenta
um elevado número de elementos indesejáveis.
      Portanto foram determinadas algumas regras para a filtragem dos
componentes da imagem. Estas regras estão listadas a seguir:      \newline
      1. componentes cuja a área não está nos limites desejáveis. \newline
      2. componentes cujo o perimetro não está nos limites desejáveis.\newline



\subsection{Extração de características}

A base de dados é composta por dois arquivos. O arquivo de imagens, onde cada
linha do arquivo é uma imagem que sofreu o pré-processamento, e o arquivo de
identificadores, onde cada linha se refere ao valor do caractere inserido na
linha de mesmo número do arquivo de imagens.


\section{O algoritmo de vizinhos mais próximos}
O algoritmo de vizinho mais próximo foi proposto por Cover e Hart em 1966. Essa
técnica é um método de estimação de densidade bastante simples conceitualmente e
de fácil implementação.
K nearest neighbors is a simple algorithm that stores all available cases and
classifies new cases based on a similarity measure (e.g., distance functions).
KNN has been used in statistical estimation and pattern recognition already in
the beginning of 1970’s as a non-parametric technique.

A case is classified by a majority vote of its neighbors, with the case being
assigned to the class most common amongst its K nearest neighbors measured by a
distance function. If K = 1, then the case is simply assigned to the class of
its nearest neighbor.

Pseudocode is defined as a listing of sequential steps for solving a
computational problem. Pseudocode is used by computer programmers to mentally
translate each computational step into a set of programming instructions
involving various mathematical operations (addition, subtraction,
multiplication, division, power and transcendental functions,
differentiation/integration, etc.) and resources (vectors, arrays, graphics,
input/output, etc.) in order to solve an analytic problem. Following is a
listing of pseudocode for the k-nearest-neighbor classification method using
cross-validation.



initialize the $n \times n$ distance matrix $\textbf{D}$ \newline
initialize the $\Omega \times \Omega$ confusion matrix $\textbf{C}$ \newline
set $t \leftarrow 0$,
$TotAcc \leftarrow 0$, and set $NumIterations$ equal to the desired number of
 iterations (re-partitions). \newline
calculate distances between all the input samples and store in $n \times n$
matrix $\textbf{D} $. (For a large number of samples, use only the lower or
upper triangular of $\textbf{D}$ for storage since it is a square symmetric
matrix.)

for $t \leftarrow 1$ to $NumIterations$ do

set $\textbf{C} \leftarrow 0$, and $n_{total} \leftarrow 0$ .
partition the input samples into $\kappa$ equally-sized groups.
for $fold \leftarrow 1$ to $\kappa$\newline 
do assign samples in the $foldth$ partition
to testing, and use the remaining samples for training. Set the number of samples
used for testing as $n_{test}$ .\newline
set $n_{total} \leftarrow n_{total}+n_{test}$ .
for $i \leftarrow 1$ to $n_{test}$\newline 
do for test sample $\textbf{x}_i$ determine the
$k$ closest training samples based on the calculated distances.
determine $\hat{\omega}$ , the most frequent class label among the $k$ closest
training samples.
increment confusion matrix $\textbf{C}$ by 1 in element
$c_{\omega,\hat{\omega}}$ , where $\omega$ is the true and $\hat{\omega}$ the
predicted class label for test sample $\textbf{x}_i$ . If $\omega =\hat{\omega}$
then the increment of +1 will occur on the diagonal of the confusion matrix, otherwise, the increment will
occur in an off-diagonal.
determine the classification accuracy using $Acc =
\frac{\sum_j^{\Omega}c_{jj}}{n_{total}}$ where $c_{jj}$ is a diagonal element of
the confusion matrix $\textbf{C}$ .
calculate $TotAcc = TotAcc + Acc$ .
calculate $AvgAcc = TotAcc/NumIterations$


\subsection{Seleção da base de treinamento}

\subsection{Algoritmo proposto}


\section{Resultados obtidos}
\section{Conclusão}
% ----------------------------------------------------------
% Referências bibliográficas
% ----------------------------------------------------------
\begin{thebibliography}{1}

\bibitem{IEEEhowto:kopka}
[1]HERBERT SCHANTZ (1982), The History of OCR. Manchester Center, VT: Recognition Technologies Users Association.

\bibitem{IEEEhowto:kopka}
[2]THE MNIST DATABASE, http://yann.lecun.com/exdb/mnist/ An introduction to neural computing. Aleksander, I. and Morton, H. 2nd edition.

\bibitem{IEEEhowto:kopka}
[3] Deolinda M. P. Aguieiras de Lima, http://mesonpi.cat.cbpf.br/naj/redesneurais.pdf/

\bibitem{IEEEhowto:kopka}
[4]JEFF HEATON, Introduction to Neural Networks with Java, second edition.

\bibitem{IEEEhowto:kopka}
[5] Facon, Jacques; Morfologia Matemática: Teoria e Exemplos. Curitiba,
 Brasil, 1996

\bibitem{IEEEhowto:kopka}
[6] Alessandra Bussador; Localização Automática de Placas de Veículos em Fotos
Digitais Utilizando Abordagem Granulométrica. Curitiba,
 Brasil, 1996
\end{thebibliography}



\end{document}
